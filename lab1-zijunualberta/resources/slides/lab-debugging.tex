\documentclass{beamer}

\usepackage{graphicx}

\useinnertheme{rectangles}
\useoutertheme{infolines}
\usecolortheme{crane}
\usefonttheme{structurebold}

\title{Debugging in SPIM}
\author[CMPUT 229]{CMPUT 229}
\institute{University of Alberta}
\date{}

\begin{document}
\frame{\titlepage}

\frame{
  \frametitle{Outline}
  \tableofcontents
}

\section{Debugging Programs}

\frame{
  \frametitle{How Do You Debug?}

  \pause

  \begin{itemize}
    \item Breakpoints
    \item Single-stepping
    \item Inspecting variable values
  \end{itemize}
}

\section{SPIM Debugging Features}

\frame{
  \frametitle{Breakpoints in SPIM}

  \begin{itemize}
    \item Can only set breakpoints by address or global symbol name.
      \begin{itemize}
        \item You can make symbols in your program global using the
          \texttt{.globl} directive. 
      \end{itemize}
    \item Look up the address of the instruction you want to break on in the
      \texttt{Text Segments} pane in XSPIM.
    \item See all the global symbols by choosing \texttt{global symbols} in the
      \texttt{print} menu in XSPIM.
    \item Use the \texttt{p(rint)} command to show instructions in SPIM.
    \item Use the \texttt{print\_symbols} command to show global symbols in SPIM.

    \item Set a breakpoint using the \texttt{breakpoints} button in XSPIM, or
      the \texttt{breakpoint} command in SPIM.
  \end{itemize}
}

\frame{
  \frametitle{Stepping in SPIM}

  \begin{itemize}
    \item Click \texttt{step} and enter the number of instructions to execute in
      XSPIM.
    \item Use the \texttt{s(tep)} command in SPIM.
      \begin{itemize}
        \item Just hit enter to repeat the last command -- handy when
          single-stepping.
      \end{itemize}
  \end{itemize}
}

\frame{
  \frametitle{Value Inspection in SPIM}
  
  \begin{itemize}
    \item As you know, there aren't really variables in assembly language.
    \item There are registers, and memory, where we keep values.
    \item In XSPIM, the registers are listed in the top pane, and values in
      memory can be found in the \texttt{Data Segments} pane.
    \item In SPIM, use the \texttt{p(rint)} command to print values in registers
      or memory.
  \end{itemize}
}

\section{Example}

\frame{
  \frametitle{Example}

  \begin{itemize}
    \item A simple program to calculate and display the Fibonacci sequence.
    \item But it doesn't quite work ...
  \end{itemize}
}

\section*{Questions?}

\frame{\frametitle{Questions?}}

\end{document}
